\documentclass[a4paper,12pt, russian, titlepage]{article} %{extreport}

\usepackage{cmap}
\usepackage[T2A]{fontenc}
\usepackage[utf8]{inputenc}
\usepackage[russian]{babel}
\usepackage{times}
\usepackage{cite}
\usepackage{algorithm,algpseudocode}
\usepackage{pscyr}
\usepackage{mathtools}
\usepackage[pdftex]{graphicx}
\usepackage{epstopdf}
\usepackage{amssymb,amsfonts,amsmath,amsthm} 
\usepackage{indentfirst} 
\usepackage[usenames,dvipsnames]{color}
\usepackage{colortbl}
\usepackage{makecell}
\usepackage{multirow} 
\usepackage{ulem}
\usepackage{physics}
\usepackage{caption}
\usepackage{subcaption}
\usepackage{enumitem}
\usepackage{pdflscape}
\usepackage{float}
\usepackage{listings}
\usepackage{pdflscape}
\usepackage{blindtext}
\usepackage[12pt]{extsizes}
\usepackage{float}
\usepackage{xcolor}
% Поля страницы
\usepackage{geometry}
\usepackage[unicode]{hyperref}
\geometry{right=1cm}
\geometry{top=2cm}
\geometry{bottom=2cm}
% КОНЕЦ: Поля страницы

\hypersetup{pdfborder={0 0 0}}
\hyphenpenalty=10000
\tolerance=9999
\binoppenalty=10000
\relpenalty=10000
\parindent=1cm
\numberwithin{equation}{section}
\abovecaptionskip=6pt
\belowcaptionskip=6pt
\abovedisplayskip=0pt
\belowdisplayskip=0pt
\abovedisplayshortskip=0pt
\belowdisplayshortskip=0pt


\makeatletter
\newcommand{\l@abcd}[2]{\hbox to\textwidth{#1\dotfill #2}}

\makeatother


\newcount\colveccount
\newcommand*\colvec[1]{
        \global\colveccount#1
        \begin{pmatrix}
        \colvecnext
}
\def\colvecnext#1{
        #1
        \global\advance\colveccount-1
        \ifnum\colveccount>0
                \\
                \expandafter\colvecnext
        \else
                \end{pmatrix}
        \fi
}

\newcommand{\bd}{\textbackslash\textbackslash}
\newcommand{\bs}{\textbackslash}
\newcommand{\empline}{\mbox{}\newline}
\newcommand{\OutOfContents}[1]{
    \newpage
    \begin{center}
    \textbf{\large{\MakeUppercase{#1}}}
    \end{center}
}
\newcommand{\Razdel}[1]{
    \newpage
    \begin{center}
    \textbf{\large{\MakeUppercase{#1}}}
    \end{center}
    \addcontentsline{toc}{abcd}{#1}
}
\renewcommand{\vec}[1]{\mathbf{#1}}
\newcommand{\append}[1]{
    %\newpage%
    \begin{flushright}
    \textbf{\large{\MakeUppercase{#1}}}
    \end{flushright}
    %\addcontentsline{toc}{abcd}{#1}%
}

\newcommand{\itab}[1]{\hspace{.55\textwidth}\rlap{#1}}
\newcommand{\tab}[1]{\hspace{.01\textwidth}\rlap{#1}}

\newcommand{\Abstr}[1]{
    \begin{center}
    \textbf{\large{\MakeUppercase{#1}}}
    \end{center}
}

\newcounter{figs}

\renewcommand{\thefigs}{Рисунок \arabic{figs}}
\renewcommand{\i}{\mathrm{i}}

\newcommand{\FigureCaption}[2]{
    \refstepcounter{figs}
    \hbox to \hsize {\hfil {\parbox{#2}{\small\begin{center}\vspace{-10pt}Рисунок \arabic{figs} -- #1\end{center}\vspace{-10pt}}}\hfil}
}

\newcommand{\dif}[2]{\frac{\partial{}#1}{\partial{}#2}}

\renewcommand{\baselinestretch}{1.2}
%\renewcommand{\bfdefault}{b}
\renewcommand{\rmdefault}{ftm} % Times New Roman

\makeatletter

\renewcommand{\section}{\@startsection{section}{1}{1cm}%
{3.4ex plus 0.9ex minus .2ex}{2ex plus.2ex}%
{\normalfont\large\bfseries}}

\renewcommand{\subsection}{\@startsection{subsection}{2}{1cm}%
{3.4ex plus 0.9ex minus .2ex}{2ex plus.2ex}%
{\normalfont\large\bfseries}}

\renewcommand{\subsubsection}{\@startsection{subsubsection}{3}{1cm}%
{3.4ex plus 0.9ex minus .2ex}{2ex plus.2ex}%
{\normalfont\large\bfseries}}

\renewcommand{\l@section}{\@dottedtocline{1}{0.39cm}{1em}}
\renewcommand{\l@subsection}{\@dottedtocline{2}{0.78cm}{1.8em}}
\renewcommand{\l@subsubsection}{\@dottedtocline{3}{1.16cm}{2.5em}}

\renewcommand{\p@figure}{Рисунок }
\renewcommand{\p@table}{Таблица }


\renewcommand{\@pnumwidth}{1em}
\renewcommand{\@dotsep}{2}
\renewcommand{\@biblabel}[1]{#1.\hfill}

\renewcommand{\@makecaption}[3]{%

\vspace{\abovecaptionskip}%			
\sbox{\@tempboxa}{#1 -- #2}
\small #1 -- #2\par%
\vspace{\belowcaptionskip}}
\newcommand{\maindef}[1]{\textup{\textbf{#1}}}
\makeatother
\newtheorem{theorem}{Теорема}
\numberwithin{theorem}{subsection}
\newtheorem{definition}{Определение}
\numberwithin{definition}{subsection}
\newtheorem{proposition}{Утверждение}
\numberwithin{proposition}{subsection}

\begin{document}
\section{Постановка}
Рассматривается задача 
\begin{equation}\label{eq:main1}
\frac{1}{r^2}\frac{\partial}{\partial r}\left(\frac{\partial u(r)}{\partial r}  \right) = f(r), \quad r \in \Omega =  \mathbb{R}_{\ge 0},
\end{equation}
\begin{equation}\label{eq:main2}
\lim_{r \to 0}r^2 \frac{\partial u(r)}{\partial r} = 0, \quad \lim_{r \to \infty} u(r) = 0.
\end{equation} \par
Обозначим $\mathcal{T}_h$ разбиение $\Omega$ (одним из элементов разбиения является множество $K_\infty = [a, \infty), \, a \in \Omega$), так что
\begin{equation}
\Omega = \bigcup_{K \in \mathcal{T}_h} K
\end{equation}
С помощью $\mathcal{F}_h$ обозначим все поверхности элементов $K \in \mathcal{T}_h$. Далее, определим множество граничных поверхностей
\begin{equation}
\mathcal{F}_h^B = \{\Gamma \in \mathcal{F}_h; \Gamma \in \partial \Omega \},
\end{equation}
и внутренних
\begin{equation}
\mathcal{F}_h^I =  \mathcal{F}_h \setminus \mathcal{F}_h^B.
\end{equation}
Для подхода к численному решению задачи (\ref{eq:main1}-\ref{eq:main2}), для начала рассмотрим билинейную форму
\begin{gather}\label{eq:main_weak}
A_h(u, v) = (f, v), \\
-\sum_{K \in \mathcal{T}_h} \int_K r^2 \frac{\partial u}{\partial r} \frac{\partial v}{\partial r} \, dx + \sum_{\Gamma \in \mathcal{F}^{I}_h} \int_\Gamma r^2 \left(\vec{n} \left \langle \frac{\partial u}{\partial r} \right\rangle [v] - \vec{n}  \left\langle \frac{\partial v}{\partial r} \right\rangle [u] \right) dS = \int_\Omega r^2 f v \, dx, 
\end{gather} 
действующую в пространстве
\begin{equation}
H^k(\Omega, \mathcal{T}_h) = \{v \in L^2(\Omega); v|_K\in H^k(K), \forall K \in \mathcal{T}_h \},
\end{equation}
с нормой
\begin{equation}
\norm{v}_{H^k(\Omega, \mathcal{T}_h)} = \left(\sum_{K \in \mathcal{T}_h} \norm{v}^2_{H^k(K)} \right)^{1/2}.
\end{equation}\par
Приближенное решение будем искать в конечном пространстве
\begin{equation}\label{eq:finite_subspace}
S_{hp} = \{v \in L^2(\Omega); v|_K \in P_p(K)\,  \forall K \in \mathcal{T}_h \},
\end{equation}
которое является подпространством $H^k(\Omega, \mathcal{T}_h)$. $P_p(K)$ в (\ref{eq:finite_subspace}) обозначает пространство полиномов степени $\le p$ на $K$. \par
В качестве базисных функций на каждой из областей решения используются полиномы Лагранжа, построенные на точках ЧГЛ (Чебышева-Гаусса-Лобатто)
\begin{equation}
x_n = - \cos \left(\frac{n \pi}{N}\right), \quad n = 0, 1, ..., N,
\end{equation}
отображенных с отрезка $[-1, 1]$ на $[a, \infty)$ или $[a, b]$ одной из двух функций: $m_a(x)$ либо $m_a^b(x)$, соответственно
\begin{equation}
m_a(x) = \frac{1 + x}{1 - x} + a, \quad m_a^b(x) = \frac{1}{2}(b - a) x + \frac{1}{2}(b + a).
\end{equation}
Таким образом вычисление значений базисных функций состоит из двух шагов:
\begin{itemize}
\item С помощью обратных функций к $m_a(x)$, $m_a^b(x)$ точки с произвольного множества переносятся на $K_r = [-1, 1]$.
\item Производится барицентрическая интерполяция с полученными точками на $K_r$.
\end{itemize}
При интегрировании используются формулы Гаусса-трапеций, которые позволяют избежать сложностей, связанных с вычислением подынтегральной функции в крайних точках.
\section{Анализ численного метода}
%Билинейная форма $A_h$ из (\ref{eq:main_weak}) удовлетворяет неравенствам/равенствам
%\begin{equation}
%|A_h(u, v)| \le \sum_{K \in \mathcal{T}_h} \int_K r^2|\frac{du}{dr} e^{-\mu r}\frac{dv}{dr}e^{\mu r}| \, dr + \sum_{x \in \mathcal{F}^I_h} r^2 |\langle \frac{du}{dr}\rangle e^{-\mu r} [v] e^{\mu r}| \big|_{r=x}+ \sum_{x \in \mathcal{F}^I_h} r^2 |\langle \frac{dv}{dr}\rangle e^{\mu r} [u]e^{-\mu r} | \big|_{r=x},
%%|A_h(u, v)| \le |u|_{H^1(\Omega, \mathcal{T}_h)} |v|_{H^1(\Omega, \mathcal{T}_h)} + \sum_{\Gamma \in \mathcal{F}^{I}_h} \int_\Gamma r^2 \left(\vec{n} \left %\langle \frac{\partial u}{\partial r} \right\rangle [v] - \vec{n}  \left\langle \frac{\partial v}{\partial r} \right\rangle [u] \right) dS
%\end{equation}
\begin{equation}
|A_h(u, v)| \le \sum_{K \in \mathcal{T}_h} \int_K r^2|\frac{du}{dr} \frac{dv}{dr}| \, dr + \sum_{x \in \mathcal{F}^I_h} r^2 |\langle \frac{du}{dr}\rangle [v]| \big|_{r=x}+ \sum_{x \in \mathcal{F}^I_h} r^2 |\langle \frac{dv}{dr}\rangle [u] | \big|_{r=x},
%|A_h(u, v)| \le |u|_{H^1(\Omega, \mathcal{T}_h)} |v|_{H^1(\Omega, \mathcal{T}_h)} + \sum_{\Gamma \in \mathcal{F}^{I}_h} \int_\Gamma r^2 \left(\vec{n} \left %\langle \frac{\partial u}{\partial r} \right\rangle [v] - \vec{n}  \left\langle \frac{\partial v}{\partial r} \right\rangle [u] \right) dS
\end{equation}
из неравенства Коши для суммы и для интегралов следует
\begin{equation}
\sum_{K \in \mathcal{T}_h} \int_K r^2|\frac{du}{dr}\frac{dv}{dr}| \, dr \le |u|_{H^1_{r^2}(\Omega, \mathcal{T}_h)} |v|_{H^1_{r^2}(\Omega, \mathcal{T}_h)}.
\end{equation} \par
Для двух других слагаемых снова используем формулу Коши для произведения функций и получаем
\begin{equation}
|A_h(u, v)| \le |u|_{H^1_{r^2}(\Omega, \mathcal{T}_h)} |v|_{H^1_{r^2}(\Omega, \mathcal{T}_h)} + \sum_{F^I_h} r^2  \langle \frac{du}{dr}\rangle^2 \sum_{F^I_h} r^2 [v]^2 + \sum_{F^I_h} r^2  \langle \frac{dv}{dr}\rangle^2 \sum_{F^I_h} r^2 [u]^2.
\end{equation}\par
Используя {\color{red}неизвестное числовое  неравенство} приходим к произведению норм
\begin{equation}
|A_h(u, v)| \le \left( |u|^2_{H^1} +  \sum_{F^I_h} r^2 \left( \langle \frac{du}{dr}\rangle^2 + [u]^2 \right) \right)^{1/2}  \left(|v|^2_{H^1} + \sum_{F^I_h} r^2 \left( \langle \frac{dv}{dr}\rangle^2 + [v]^2 \right) \right)^{1/2},
\end{equation}
 где $H_1 = H^1_{r^2}(\Omega, \mathcal{T}_h)$.
Таким образом получили необходимую норму
\begin{equation}
\norm{u}_{H^1_h} = \left( |u|^2_{H^1_{r^2}(\Omega, \mathcal{T}_h)} +  \sum_{F^I_h} r^2 \left( \langle \frac{du}{dr}\rangle^2 + [u]^2 \right) \right)^{1/2},
\end{equation}
и соответствующее неравенство
\begin{equation}
A_h(u, v)| \le \norm{u}_{H^1_h} \norm{v}_{H^1_h}.
\end{equation}\par

 
В итоге всех манипуляций получаем
\begin{equation}
e_h = u_h - u = \xi + \eta, \quad \xi = u_h - \Pi_h u \ \in V_h, \quad \eta = \Pi_h u - u \in W_h.
\end{equation}
\begin{equation}\label{eq:galerkin_orth}
A(e_h, v_h) = 0, \quad v_h \in V_h,
\end{equation}
Из (\ref{eq:galerkin_orth}) следует
\begin{equation}
A_h(e_h, \xi) = A_h(\xi, \xi) + A_h(\eta, \xi) = 0.
\end{equation}
\begin{equation}
A_h(v, v) = |v|_{H^1(\Omega, \mathcal{T}_h)}^2
\end{equation}
\begin{equation}
\norm{\xi}^2 \le A_h(\xi, \xi) = -A_h(\eta, \xi) \le \norm{\eta} \norm{\xi}
\end{equation}
откуда  следует
\begin{equation}
\norm{\xi} \le \norm{\eta}
\end{equation}
и в итоге
\begin{equation}
\norm{e_h} \le \norm{\xi} + \norm{\eta} \le 2\norm{\eta}
\end{equation}
\subsection{По поводу погрешности интерполяции}
Нужно получить какую-либо оценку для 
\begin{equation}
\norm{v - \Pi_h v}_{W_h} \le ??
\end{equation}
\section{Аналитическое решение}
Правая часть $f = \exp(-r)$. \par
Общее решение
\begin{equation}
u_g(r) = \frac{e^{-r} (2+r) - C_1 + r C_2}{r}
\end{equation}
Пределы
\begin{gather}
\lim_{r \to \infty} u(r) = C_2, \\
\lim_{r \to 0} r^2 u'(r) = C_1 - 2.
\end{gather}
Частное решение тогда имеет вид
\begin{equation}
u(r) = e^{-r} + \frac{2 e^{-r}}{r} - \frac{2}{r} 
\end{equation}
%\begin{equation}
%|\pi_{K,p}v - v|_H^q(K) \le 
%\end{equation}


\end{document}
